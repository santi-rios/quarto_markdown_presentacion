% Options for packages loaded elsewhere
\PassOptionsToPackage{unicode}{hyperref}
\PassOptionsToPackage{hyphens}{url}
\PassOptionsToPackage{dvipsnames,svgnames,x11names}{xcolor}
%
\documentclass[
  letterpaper,
  DIV=11,
  numbers=noendperiod]{scrartcl}

\usepackage{amsmath,amssymb}
\usepackage{iftex}
\ifPDFTeX
  \usepackage[T1]{fontenc}
  \usepackage[utf8]{inputenc}
  \usepackage{textcomp} % provide euro and other symbols
\else % if luatex or xetex
  \usepackage{unicode-math}
  \defaultfontfeatures{Scale=MatchLowercase}
  \defaultfontfeatures[\rmfamily]{Ligatures=TeX,Scale=1}
\fi
\usepackage{lmodern}
\ifPDFTeX\else  
    % xetex/luatex font selection
\fi
% Use upquote if available, for straight quotes in verbatim environments
\IfFileExists{upquote.sty}{\usepackage{upquote}}{}
\IfFileExists{microtype.sty}{% use microtype if available
  \usepackage[]{microtype}
  \UseMicrotypeSet[protrusion]{basicmath} % disable protrusion for tt fonts
}{}
\makeatletter
\@ifundefined{KOMAClassName}{% if non-KOMA class
  \IfFileExists{parskip.sty}{%
    \usepackage{parskip}
  }{% else
    \setlength{\parindent}{0pt}
    \setlength{\parskip}{6pt plus 2pt minus 1pt}}
}{% if KOMA class
  \KOMAoptions{parskip=half}}
\makeatother
\usepackage{xcolor}
\setlength{\emergencystretch}{3em} % prevent overfull lines
\setcounter{secnumdepth}{5}
% Make \paragraph and \subparagraph free-standing
\ifx\paragraph\undefined\else
  \let\oldparagraph\paragraph
  \renewcommand{\paragraph}[1]{\oldparagraph{#1}\mbox{}}
\fi
\ifx\subparagraph\undefined\else
  \let\oldsubparagraph\subparagraph
  \renewcommand{\subparagraph}[1]{\oldsubparagraph{#1}\mbox{}}
\fi

\usepackage{color}
\usepackage{fancyvrb}
\newcommand{\VerbBar}{|}
\newcommand{\VERB}{\Verb[commandchars=\\\{\}]}
\DefineVerbatimEnvironment{Highlighting}{Verbatim}{commandchars=\\\{\}}
% Add ',fontsize=\small' for more characters per line
\usepackage{framed}
\definecolor{shadecolor}{RGB}{241,243,245}
\newenvironment{Shaded}{\begin{snugshade}}{\end{snugshade}}
\newcommand{\AlertTok}[1]{\textcolor[rgb]{0.68,0.00,0.00}{#1}}
\newcommand{\AnnotationTok}[1]{\textcolor[rgb]{0.37,0.37,0.37}{#1}}
\newcommand{\AttributeTok}[1]{\textcolor[rgb]{0.40,0.45,0.13}{#1}}
\newcommand{\BaseNTok}[1]{\textcolor[rgb]{0.68,0.00,0.00}{#1}}
\newcommand{\BuiltInTok}[1]{\textcolor[rgb]{0.00,0.23,0.31}{#1}}
\newcommand{\CharTok}[1]{\textcolor[rgb]{0.13,0.47,0.30}{#1}}
\newcommand{\CommentTok}[1]{\textcolor[rgb]{0.37,0.37,0.37}{#1}}
\newcommand{\CommentVarTok}[1]{\textcolor[rgb]{0.37,0.37,0.37}{\textit{#1}}}
\newcommand{\ConstantTok}[1]{\textcolor[rgb]{0.56,0.35,0.01}{#1}}
\newcommand{\ControlFlowTok}[1]{\textcolor[rgb]{0.00,0.23,0.31}{#1}}
\newcommand{\DataTypeTok}[1]{\textcolor[rgb]{0.68,0.00,0.00}{#1}}
\newcommand{\DecValTok}[1]{\textcolor[rgb]{0.68,0.00,0.00}{#1}}
\newcommand{\DocumentationTok}[1]{\textcolor[rgb]{0.37,0.37,0.37}{\textit{#1}}}
\newcommand{\ErrorTok}[1]{\textcolor[rgb]{0.68,0.00,0.00}{#1}}
\newcommand{\ExtensionTok}[1]{\textcolor[rgb]{0.00,0.23,0.31}{#1}}
\newcommand{\FloatTok}[1]{\textcolor[rgb]{0.68,0.00,0.00}{#1}}
\newcommand{\FunctionTok}[1]{\textcolor[rgb]{0.28,0.35,0.67}{#1}}
\newcommand{\ImportTok}[1]{\textcolor[rgb]{0.00,0.46,0.62}{#1}}
\newcommand{\InformationTok}[1]{\textcolor[rgb]{0.37,0.37,0.37}{#1}}
\newcommand{\KeywordTok}[1]{\textcolor[rgb]{0.00,0.23,0.31}{#1}}
\newcommand{\NormalTok}[1]{\textcolor[rgb]{0.00,0.23,0.31}{#1}}
\newcommand{\OperatorTok}[1]{\textcolor[rgb]{0.37,0.37,0.37}{#1}}
\newcommand{\OtherTok}[1]{\textcolor[rgb]{0.00,0.23,0.31}{#1}}
\newcommand{\PreprocessorTok}[1]{\textcolor[rgb]{0.68,0.00,0.00}{#1}}
\newcommand{\RegionMarkerTok}[1]{\textcolor[rgb]{0.00,0.23,0.31}{#1}}
\newcommand{\SpecialCharTok}[1]{\textcolor[rgb]{0.37,0.37,0.37}{#1}}
\newcommand{\SpecialStringTok}[1]{\textcolor[rgb]{0.13,0.47,0.30}{#1}}
\newcommand{\StringTok}[1]{\textcolor[rgb]{0.13,0.47,0.30}{#1}}
\newcommand{\VariableTok}[1]{\textcolor[rgb]{0.07,0.07,0.07}{#1}}
\newcommand{\VerbatimStringTok}[1]{\textcolor[rgb]{0.13,0.47,0.30}{#1}}
\newcommand{\WarningTok}[1]{\textcolor[rgb]{0.37,0.37,0.37}{\textit{#1}}}

\providecommand{\tightlist}{%
  \setlength{\itemsep}{0pt}\setlength{\parskip}{0pt}}\usepackage{longtable,booktabs,array}
\usepackage{calc} % for calculating minipage widths
% Correct order of tables after \paragraph or \subparagraph
\usepackage{etoolbox}
\makeatletter
\patchcmd\longtable{\par}{\if@noskipsec\mbox{}\fi\par}{}{}
\makeatother
% Allow footnotes in longtable head/foot
\IfFileExists{footnotehyper.sty}{\usepackage{footnotehyper}}{\usepackage{footnote}}
\makesavenoteenv{longtable}
\usepackage{graphicx}
\makeatletter
\def\maxwidth{\ifdim\Gin@nat@width>\linewidth\linewidth\else\Gin@nat@width\fi}
\def\maxheight{\ifdim\Gin@nat@height>\textheight\textheight\else\Gin@nat@height\fi}
\makeatother
% Scale images if necessary, so that they will not overflow the page
% margins by default, and it is still possible to overwrite the defaults
% using explicit options in \includegraphics[width, height, ...]{}
\setkeys{Gin}{width=\maxwidth,height=\maxheight,keepaspectratio}
% Set default figure placement to htbp
\makeatletter
\def\fps@figure{htbp}
\makeatother

\KOMAoption{captions}{tableheading}
\makeatletter
\@ifpackageloaded{tcolorbox}{}{\usepackage[skins,breakable]{tcolorbox}}
\@ifpackageloaded{fontawesome5}{}{\usepackage{fontawesome5}}
\definecolor{quarto-callout-color}{HTML}{909090}
\definecolor{quarto-callout-note-color}{HTML}{0758E5}
\definecolor{quarto-callout-important-color}{HTML}{CC1914}
\definecolor{quarto-callout-warning-color}{HTML}{EB9113}
\definecolor{quarto-callout-tip-color}{HTML}{00A047}
\definecolor{quarto-callout-caution-color}{HTML}{FC5300}
\definecolor{quarto-callout-color-frame}{HTML}{acacac}
\definecolor{quarto-callout-note-color-frame}{HTML}{4582ec}
\definecolor{quarto-callout-important-color-frame}{HTML}{d9534f}
\definecolor{quarto-callout-warning-color-frame}{HTML}{f0ad4e}
\definecolor{quarto-callout-tip-color-frame}{HTML}{02b875}
\definecolor{quarto-callout-caution-color-frame}{HTML}{fd7e14}
\makeatother
\makeatletter
\@ifpackageloaded{caption}{}{\usepackage{caption}}
\AtBeginDocument{%
\ifdefined\contentsname
  \renewcommand*\contentsname{Table of contents}
\else
  \newcommand\contentsname{Table of contents}
\fi
\ifdefined\listfigurename
  \renewcommand*\listfigurename{List of Figures}
\else
  \newcommand\listfigurename{List of Figures}
\fi
\ifdefined\listtablename
  \renewcommand*\listtablename{List of Tables}
\else
  \newcommand\listtablename{List of Tables}
\fi
\ifdefined\figurename
  \renewcommand*\figurename{Figure}
\else
  \newcommand\figurename{Figure}
\fi
\ifdefined\tablename
  \renewcommand*\tablename{Table}
\else
  \newcommand\tablename{Table}
\fi
}
\@ifpackageloaded{float}{}{\usepackage{float}}
\floatstyle{ruled}
\@ifundefined{c@chapter}{\newfloat{codelisting}{h}{lop}}{\newfloat{codelisting}{h}{lop}[chapter]}
\floatname{codelisting}{Listing}
\newcommand*\listoflistings{\listof{codelisting}{List of Listings}}
\makeatother
\makeatletter
\makeatother
\makeatletter
\@ifpackageloaded{caption}{}{\usepackage{caption}}
\@ifpackageloaded{subcaption}{}{\usepackage{subcaption}}
\makeatother
\ifLuaTeX
  \usepackage{selnolig}  % disable illegal ligatures
\fi
\usepackage{bookmark}

\IfFileExists{xurl.sty}{\usepackage{xurl}}{} % add URL line breaks if available
\urlstyle{same} % disable monospaced font for URLs
\hypersetup{
  pdftitle={Tutorial-Ejercicio para Rmarkdown/Quarto},
  pdfauthor={Garcia Rios Santiago; Mariana Guerrero Osornio},
  colorlinks=true,
  linkcolor={blue},
  filecolor={Maroon},
  citecolor={Blue},
  urlcolor={Blue},
  pdfcreator={LaTeX via pandoc}}

\title{Tutorial-Ejercicio para Rmarkdown/Quarto}
\author{Garcia Rios Santiago \and Mariana Guerrero Osornio}
\date{2024-04-10}

\begin{document}
\maketitle

\renewcommand*\contentsname{Table of contents}
{
\hypersetup{linkcolor=}
\setcounter{tocdepth}{3}
\tableofcontents
}
\begin{itemize}
\tightlist
\item[$\square$]
  Crear nuevo documento \texttt{html} en RMarkdown/Quarto
\end{itemize}

\texttt{File\ -\textgreater{}\ New\ file\ -\textgreater{}\ R\ Markdown/Quarto}

\begin{tcolorbox}[enhanced jigsaw, colbacktitle=quarto-callout-note-color!10!white, opacitybacktitle=0.6, bottomrule=.15mm, colframe=quarto-callout-note-color-frame, opacityback=0, toptitle=1mm, breakable, toprule=.15mm, left=2mm, bottomtitle=1mm, leftrule=.75mm, arc=.35mm, title=\textcolor{quarto-callout-note-color}{\faInfo}\hspace{0.5em}{Note}, titlerule=0mm, rightrule=.15mm, coltitle=black, colback=white]

Es probable que necesites instalar
\texttt{install.packages("rmarkdown")} y/o
\texttt{install.packages("quarto")}

\end{tcolorbox}

\includegraphics{./figuras/new-rmarkdown.gif}

\begin{itemize}
\tightlist
\item[$\square$]
  Renderizar el documento
\end{itemize}

\texttt{ctrl\ +\ shift\ +\ k} o con:

\includegraphics{./figuras/knitting.gif}

\begin{itemize}
\tightlist
\item[$\square$]
  Ver output
\end{itemize}

\includegraphics{./figuras/preview.gif}

\begin{itemize}
\tightlist
\item[$\square$]
  Editar YAML y volver a renderizar
\end{itemize}

\begin{tcolorbox}[enhanced jigsaw, colbacktitle=quarto-callout-tip-color!10!white, opacitybacktitle=0.6, bottomrule=.15mm, colframe=quarto-callout-tip-color-frame, opacityback=0, toptitle=1mm, breakable, toprule=.15mm, left=2mm, bottomtitle=1mm, leftrule=.75mm, arc=.35mm, title=\textcolor{quarto-callout-tip-color}{\faLightbulb}\hspace{0.5em}{Tip}, titlerule=0mm, rightrule=.15mm, coltitle=black, colback=white]

El YAML usado para este ejemplo es el siguiente:

\begin{verbatim}
---
title: "Tutorial-Ejercicio para Rmarkdown/Quarto"
author:
  - name: Garcia Rios Santiago
    affiliations:
      - name: UNAM
        url: https://www.unam.mx/
  - name: Mariana Guerrero Osornio
    affiliations:
      - name: UNAM
        url: https://www.unam.mx/
format: 
  html:
    page-layout: full # ocupar toda la página
    lang: es  # figure, note, warning, code
    embed-resources: true # self-contained file
    # code-fold: true # retraer código
    # code-summary: "Mostrar código"
    other-links:
      - text: Ver presentación en línea
        href: https://santi-rios.github.io/quarto-iframes/#/section
    code-links:
      - text: Código de la presentación
        icon: github
        href: https://github.com/santi-rios/quarto-iframes/blob/main/index.qmd
date: "today"
execute:
  echo: true  
  warning: false
toc: true
editor: source
# bibliography: references.bib  
number-sections: true
---
\end{verbatim}

\end{tcolorbox}

\includegraphics{./figuras/style-markdown.gif}

\begin{itemize}
\tightlist
\item[$\square$]
  Insertar código en R y ejecutarlo dentro del documento
\end{itemize}

\includegraphics{./figuras/insert-rchunk.gif}
\includegraphics{./figuras/run-chunk.gif}

\begin{itemize}
\tightlist
\item[$\square$]
  Crea Encabezados
\end{itemize}

\begin{tcolorbox}[enhanced jigsaw, colbacktitle=quarto-callout-tip-color!10!white, opacitybacktitle=0.6, bottomrule=.15mm, colframe=quarto-callout-tip-color-frame, opacityback=0, toptitle=1mm, breakable, toprule=.15mm, left=2mm, bottomtitle=1mm, leftrule=.75mm, arc=.35mm, title=\textcolor{quarto-callout-tip-color}{\faLightbulb}\hspace{0.5em}{Tip}, titlerule=0mm, rightrule=.15mm, coltitle=black, colback=white]

\begin{verbatim}
# Mi Primer Encabezado

## S
\end{verbatim}

\end{tcolorbox}

\begin{itemize}
\tightlist
\item[$\square$]
  Crea texto en negritas y cursiva.
\end{itemize}

\begin{tcolorbox}[enhanced jigsaw, colbacktitle=quarto-callout-tip-color!10!white, opacitybacktitle=0.6, bottomrule=.15mm, colframe=quarto-callout-tip-color-frame, opacityback=0, toptitle=1mm, breakable, toprule=.15mm, left=2mm, bottomtitle=1mm, leftrule=.75mm, arc=.35mm, title=\textcolor{quarto-callout-tip-color}{\faLightbulb}\hspace{0.5em}{Tip}, titlerule=0mm, rightrule=.15mm, coltitle=black, colback=white]

\begin{verbatim}
Este es un texto con **negrita** y con *cursiva*.
\end{verbatim}

\end{tcolorbox}

\begin{itemize}
\tightlist
\item[$\square$]
  Crea una lista
\end{itemize}

\begin{tcolorbox}[enhanced jigsaw, colbacktitle=quarto-callout-tip-color!10!white, opacitybacktitle=0.6, bottomrule=.15mm, colframe=quarto-callout-tip-color-frame, opacityback=0, toptitle=1mm, breakable, toprule=.15mm, left=2mm, bottomtitle=1mm, leftrule=.75mm, arc=.35mm, title=\textcolor{quarto-callout-tip-color}{\faLightbulb}\hspace{0.5em}{Tip}, titlerule=0mm, rightrule=.15mm, coltitle=black, colback=white]

\begin{verbatim}
- Primer ítem
- Segundo ítem
- Tercer ítem

1. Primer paso
2. Segundo paso
3. Tercer paso
\end{verbatim}

\end{tcolorbox}

\begin{itemize}
\tightlist
\item[$\square$]
  Crea un enlace a la pagina https://www.r-project.org que diga ``Página
  oficial de R''
\end{itemize}

\begin{tcolorbox}[enhanced jigsaw, colbacktitle=quarto-callout-tip-color!10!white, opacitybacktitle=0.6, bottomrule=.15mm, colframe=quarto-callout-tip-color-frame, opacityback=0, toptitle=1mm, breakable, toprule=.15mm, left=2mm, bottomtitle=1mm, leftrule=.75mm, arc=.35mm, title=\textcolor{quarto-callout-tip-color}{\faLightbulb}\hspace{0.5em}{Tip}, titlerule=0mm, rightrule=.15mm, coltitle=black, colback=white]

\texttt{{[}Página\ oficial\ de\ R{]}(https://www.r-project.org)}

\end{tcolorbox}

\begin{itemize}
\tightlist
\item[$\square$]
  Remueve la opción para que el siguiente código SI se ejecute y observa
  lo que pasa con \texttt{@fig-airquality} cuando se renderiza el
  documento.
\end{itemize}

\begin{tcolorbox}[enhanced jigsaw, colbacktitle=quarto-callout-warning-color!10!white, opacitybacktitle=0.6, bottomrule=.15mm, colframe=quarto-callout-warning-color-frame, opacityback=0, toptitle=1mm, breakable, toprule=.15mm, left=2mm, bottomtitle=1mm, leftrule=.75mm, arc=.35mm, title=\textcolor{quarto-callout-warning-color}{\faExclamationTriangle}\hspace{0.5em}{Warning}, titlerule=0mm, rightrule=.15mm, coltitle=black, colback=white]

Este paso solo funciona con quarto.

\end{tcolorbox}

\begin{tcolorbox}[enhanced jigsaw, colbacktitle=quarto-callout-note-color!10!white, opacitybacktitle=0.6, bottomrule=.15mm, colframe=quarto-callout-note-color-frame, opacityback=0, toptitle=1mm, breakable, toprule=.15mm, left=2mm, bottomtitle=1mm, leftrule=.75mm, arc=.35mm, title=\textcolor{quarto-callout-note-color}{\faInfo}\hspace{0.5em}{Note}, titlerule=0mm, rightrule=.15mm, coltitle=black, colback=white]

\begin{verbatim}
#| eval: false
#| label: fig-airquality
#| fig-cap: "Niveles de ozono y temperatura."

library(ggplot2)

ggplot(airquality, aes(Temp, Ozone)) + 
  geom_point() + 
  geom_smooth(method = "lm")
\end{verbatim}

\end{tcolorbox}

En la \textbf{?@fig-airquality} observamos el impacto de la temperatura
sobre los niveles de ozono.

\begin{Shaded}
\begin{Highlighting}[]
\FunctionTok{library}\NormalTok{(ggplot2)}

\FunctionTok{ggplot}\NormalTok{(airquality, }\FunctionTok{aes}\NormalTok{(Temp, Ozone)) }\SpecialCharTok{+} 
  \FunctionTok{geom\_point}\NormalTok{() }\SpecialCharTok{+} 
  \FunctionTok{geom\_smooth}\NormalTok{(}\AttributeTok{method =} \StringTok{"lm"}\NormalTok{)}
\end{Highlighting}
\end{Shaded}

\begin{itemize}
\tightlist
\item[$\square$]
  Quitarle los ` y renderizar para ver la ecuación:
\end{itemize}

\begin{tcolorbox}[enhanced jigsaw, colbacktitle=quarto-callout-note-color!10!white, opacitybacktitle=0.6, bottomrule=.15mm, colframe=quarto-callout-note-color-frame, opacityback=0, toptitle=1mm, breakable, toprule=.15mm, left=2mm, bottomtitle=1mm, leftrule=.75mm, arc=.35mm, title=\textcolor{quarto-callout-note-color}{\faInfo}\hspace{0.5em}{Note}, titlerule=0mm, rightrule=.15mm, coltitle=black, colback=white]

Probablemente se necesite \texttt{tinytex::install\_tinytex()}

\end{tcolorbox}

Estimador mínimo cuadrático de la varianza:

\begin{verbatim}
$$
S^2=\frac{\sum\limits_{i=1}^{n}{(X_i-\bar{X})^2}}{n-1}
$$
\end{verbatim}

\begin{itemize}
\item[$\square$]
  Explora el editor Visual
\item[$\square$]
  Crea un formato pdf y docx y renderizalos
\end{itemize}

\begin{verbatim}
---
title: ""
author: ""
format:
  html: 
    code-fold: true
  pdf:
    geometry: 
      - top=30mm
      - left=30mm
  docx: default
---
\end{verbatim}



\end{document}
